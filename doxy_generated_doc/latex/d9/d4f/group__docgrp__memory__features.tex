\hypertarget{group__docgrp__memory__features}{}\section{Memory features}
\label{group__docgrp__memory__features}\index{Memory features@{Memory features}}


Module describing memory features.  


Module describing memory features. 

\subsubsection*{In-\/core R\+AM partitions}

The R\+AM memory of each core is divided in\+:


\begin{DoxyEnumerate}
\item Kernel code. 
\item Stack. 
\item Module code. See \hyperlink{group__docgrp__loading__features}{Loading features} 
\item Dynamically allocatable memory. 
\end{DoxyEnumerate}

The sizes and location are configurable in the G\+NU link script. See \hyperlink{group__docgrp__linking__features}{Linking features}.

\subsubsection*{Dynamic memory}

The in-\/core dynamic memory is handled in a two fold way\+:

\paragraph*{First time heap allocation}

The first time an object is allocated the library calls external code (off-\/core) of the \href{https://github.com/dimonomid/umm_malloc}{\tt umm\+\_\+malloc} allocator.

\paragraph*{Reusing objects}

Objects are never really expected to be freed (umm\+\_\+free call) under normal use of the library. They just get acquired (see for example \hyperlink{classcell_ad710b11caf02e0a3d2e281cb0d46a44b}{cell\+::acquire}) ; and when finisehd using them, they should call the \hyperlink{classagent_a10d8b14221d243aa3906c2d65d3fe7e8}{agent\+::release} method to get inited and put in a one-\/per-\/class double linked list (grip) of available objects (see \hyperlink{classagent_af4bd30d8eb23b3f09e115e4efda993da}{agent\+::get\+\_\+available} virtual method).

The one-\/per-\/class acquire and the \hyperlink{classagent_a10d8b14221d243aa3906c2d65d3fe7e8}{agent\+::release} are small footprint methods runned from in-\/core memory (as opposed to the umm\+\_\+malloc functions) that can be used in the user\textquotesingle{}s high performance code.

This double folded approach to memory managment also helps to avoid bad referencing of objects because a \hyperlink{classcell}{cell} might have been \hyperlink{classagent_a10d8b14221d243aa3906c2d65d3fe7e8}{agent\+::release} d but the reference is still valid, so the code will not behave as expected (there is an error) but it will not hang.

Functions like \hyperlink{classcell_ad09500ef03d2ccb60de17f189ba59b9e}{cell\+::separate} \hyperlink{classcell_ad710b11caf02e0a3d2e281cb0d46a44b}{cell\+::acquire} are provided for base classes but the user is expected to use macro \hyperlink{cell_8hh_ab43adc7961f0b5b06463b122ae598993}{M\+C\+K\+\_\+\+D\+E\+C\+L\+A\+R\+E\+\_\+\+M\+E\+M\+\_\+\+M\+E\+T\+H\+O\+DS} and macro \hyperlink{cell_8hh_a9095d9f262f63984c60799e4a3b135c5}{M\+C\+K\+\_\+\+D\+E\+F\+I\+N\+E\+\_\+\+M\+E\+M\+\_\+\+M\+E\+T\+H\+O\+DS} to declare and define these functions for derived classes. The macro \hyperlink{cell_8hh_aad0dec1d661f1fdb5a2ccb767d712c75}{M\+C\+K\+\_\+\+D\+E\+F\+I\+N\+E\+\_\+\+A\+C\+Q\+U\+I\+R\+E\+\_\+\+A\+L\+L\+OC} can also be used for classes of objects that will not get \hyperlink{classagent_a10d8b14221d243aa3906c2d65d3fe7e8}{agent\+::release} d and therefore do not declare an available list.

\paragraph*{Eating Philosophers}

The classical example \href{https://en.wikipedia.org/wiki/Dining_philosophers_problem}{\tt eating philosophers} for synchronizing concurrency is given in the \hyperlink{philo_8cpp}{philo.\+cpp} example program.

~\newline
~\newline
~\newline
~\newline
 